%
% This is a borrowed LaTeX template file for lecture notes for CS267,
% Applications of Parallel Computing, UCBerkeley EECS Department.
% Now being used for CMU's 10725 Fall 2012 Optimization course
% taught by Geoff Gordon and Ryan Tibshirani.  When preparing 
% LaTeX notes for this class, please use this template.
%
% To familiarize yourself with this template, the body contains
% some examples of its use.  Look them over.  Then you can
% run LaTeX on this file.  After you have LaTeXed this file then
% you can look over the result either by printing it out with
% dvips or using xdvi. "pdflatex template.tex" should also work.
%

\documentclass[twoside]{article}
\setlength{\oddsidemargin}{0.25 in}
\setlength{\evensidemargin}{-0.25 in}
\setlength{\topmargin}{-0.6 in}
\setlength{\textwidth}{6.5 in}
\setlength{\textheight}{8.5 in}
\setlength{\headsep}{0.75 in}
\setlength{\parindent}{0 in}
\setlength{\parskip}{0.1 in}

%
% ADD PACKAGES here:
%

\usepackage{amsmath,amsfonts,graphicx,datetime}
\usepackage{polski}
\usepackage[utf8]{inputenc}

%
% The following commands set up the lecnum (lecture number)
% counter and make various numbering schemes work relative
% to the lecture number.
%
\newcounter{lecnum}
\renewcommand{\thepage}{\thelecnum-\arabic{page}}
\renewcommand{\thesection}{\thelecnum.\arabic{section}}
\renewcommand{\theequation}{\thelecnum.\arabic{equation}}
\renewcommand{\thefigure}{\thelecnum.\arabic{figure}}
\renewcommand{\thetable}{\thelecnum.\arabic{table}}

%
% The following macro is used to generate the header.
%
\newcommand{\lecture}[4]{
   \pagestyle{myheadings}
   \thispagestyle{plain}
   \newpage
   \setcounter{lecnum}{#1}
   \setcounter{page}{1}
   \noindent
   \begin{center}
   \framebox{
      \vbox{\vspace{2mm}
    \hbox to 6.28in { {\bf 10-725: Optimization
	\hfill Fall 2012} }
       \vspace{4mm}
       \hbox to 6.28in { {\Large \hfill Lecture #1: #2  \hfill} }
       \vspace{2mm}
       \hbox to 6.28in { {\it Lecturer: #3 \hfill Scribes: #4} }
      \vspace{2mm}}
   }
   \end{center}
   \markboth{Lecture #1: #2}{Lecture #1: #2}

   {\bf Note}: {\it Template.}

   \vspace*{4mm}
}
%
% Convention for citations is authors' initials followed by the year.
% For example, to cite a paper by Leighton and Maggs you would type
% \cite{LM89}, and to cite a paper by Strassen you would type \cite{S69}.
% (To avoid bibliography problems, for now we redefine the \cite command.)
% Also commands that create a suitable format for the reference list.
\renewcommand{\cite}[1]{[#1]}
\def\beginrefs{\begin{list}%
        {[\arabic{equation}]}{\usecounter{equation}
         \setlength{\leftmargin}{2.0truecm}\setlength{\labelsep}{0.4truecm}%
         \setlength{\labelwidth}{1.6truecm}}}
\def\endrefs{\end{list}}
\def\bibentry#1{\item[\hbox{[#1]}]}

%Use this command for a figure; it puts a figure in wherever you want it.
%usage: \fig{NUMBER}{SPACE-IN-INCHES}{CAPTION}
\newcommand{\fig}[3]{
			\vspace{#2}
			\begin{center}
			Figure \thelecnum.#1:~#3
			\end{center}
	}
% Use these for theorems, lemmas, proofs, etc.
\newtheorem{theorem}{Theorem}[lecnum]
\newtheorem{lemma}[theorem]{Lemma}
\newtheorem{proposition}[theorem]{Proposition}
\newtheorem{claim}[theorem]{Claim}
\newtheorem{corollary}[theorem]{Corollary}
\newtheorem{definition}[theorem]{Definition}
\newenvironment{proof}{{\bf Proof:}}{\hfill\rule{2mm}{2mm}}

% **** IF YOU WANT TO DEFINE ADDITIONAL MACROS FOR YOURSELF, PUT THEM HERE:

\newcommand\E{\mathbb{E}}

\begin{document}
%FILL IN THE RIGHT INFO.
%\lecture{**LECTURE-NUMBER**}{**DATE**}{**LECTURER**}{**SCRIBE**}
%\footnotetext{These notes are partially based on those of Nigel Mansell.}

% **** YOUR NOTES GO HERE:

% Some general latex examples and examples making use of the
% macros follow.  
%**** IN GENERAL, BE BRIEF. LONG SCRIBE NOTES, NO MATTER HOW WELL WRITTEN,
%**** ARE NEVER READ BY ANYBODY.

\section{Wstęp do isoperimetric gap} % Don't be this informal in your notes!

!!! Pisz WYRAŹNIE

!! Prawa tablica, u góry w rogu

Powiedzieć o:
\begin{itemize}

\item Funkcja Dehna już była, zapisać tylko definicję w rogu
\item Wspomnieć, że jest sens mówić o prezentacji Dehna grupy, z dokładnością do relacji g(n) $<=$ f(C n + l) + C n + C
\item Grupy hiperboliczne mają prezentację Dehna i liniową funkcję Dehna w związku z tym.
\item Grupy automatyczne mają kwadratową funkcję Dehna.
\item Czy istnieją grupy z funkcją Dehna pomiędzy?

\end{itemize}

!! Środkowa tablica, trzecia od lewej

\begin{lemma}
Jeśli funkcja Dehna grupy G jest $o(n^2)$, to G jest hiperboliczna.
\end{lemma}

Będziemy dążyli do oszacowania z dołu funkcji Dehna przez wartość rosnącą z kwadratem długości, przy założeniu, że nie jest hiperboliczna grupa G.

Sformułujemy pomocniczy lemat o geodezyjnych przestrzeniach hiperbolicznych. Nie mam dobrych pomysłów na tłumaczenia niektórych rzeczy.

!! Tablica najbardziej po lewej

\begin{lemma}

Dla trójkąta T = ABC, mamy na boku AB punkt O, że odległość z O do AC jest równa odległości do BC. Taki punkt to bisector, zdefiniujmy jego wielkość jako b(O).

(Taki punkt istnieje z ciągłości różnicy odległości).

\end{lemma}

\begin{lemma}

Bisizem trójkąta T = ABC nazwiemy maximum b(O) dla wszystkich bisektorów O.

\end{lemma}

\begin{lemma}

Obwód wielokąta to suma długości jego boków.

\end{lemma}

! Ważna definicja

\begin{lemma}

Grubością wielkokąta nazwiemy infimum po wartościach t spełniających:

Dla każdego boku p i każdego punktu $O \in p$, istnieje inny bok $r \neq p$, że $d(O, r) \leq t$

Ten zbiór nie jest pusty, bo średnica spełnia tą definicję.
(średnica istnieje, bo pomiędzy każdymi dwoma punktmai jest ścieżka o długości krótszej niż obwód wielokąta).

Dla trójkątów to dokładnie definicja hiperboliczności (gdyby wszystkie takie t były ograniczone).

\end{lemma}

!! środkowa tablica

\begin{lemma}

{\huge 3}

Przypuśćmy, że bisizy trójkątów w przestrzeni geodezyjnej X są nieograniczone. Wtedy dla dowolnego $t_{0}$ istnieje sześciokąt P w X, którego grubość $t > t_{0}$, a jego średnica jest nie większa niż $46 t$

\end{lemma}

Tego powyżej nie będziemy dowodzić.

Należy zacząć od przypomnienia o diagramach van Kampena.
Diagram van Kampena to graf planarny, na każdej krawędzi jest słowo lub odwrotność z alfabetu. Ściany to ograniczone składowe grafu. Słowo diagramu, to słowo będące na krawędzi nieograniczonej składowej diagramu. Czytajac słowo przechodzimy  po krawędziach. Będziemy patrzeć na słowa równoważne $\epsilon$ w grupie G.

!! Środkowa tablica

\begin{lemma}

{\huge 4}

Przypuśćmy, że bisizy trójkątów w grafie Cayleya $\Gamma$ grupy G są nieograniczone (G nie jest hiperboliczne). Wtedy dla każdego $t_{0}$ istnieje w niej wielokąt o grubości $t \geq t_{0}$ i średnicy $\leq 47 t$

\end{lemma}

!! Środkowa tablica

Główny lemat jest następujący:

\begin{lemma}
 
{\huge 5}

 
Jeśli t jest grubością diagramu van Kampena, z wielkokątem p jako brzegiem, to ilość ścian jest równa przynajmniej 
\[ \frac{4t^2}{M^3} \]

gdzie M to maksimum długości relatorów.

\end{lemma}

\begin{theorem}

Jeśli funkcja Dehna jest $o(n^2)$, to G jest hiperboliczna.

\end{theorem}


Dowód: przypuśćmy, że nie jest hiperboliczna, wtedy w grafie Cayleya mamy nieograniczone bisizy trójkątów i stosuje się lemat 4. Na jego podstawie dostajemy wielokąt o grubości t ($t > t_{0}$, t dowolne) i średnicy $\leq 47t$. Odpowiada on słowu w (brzeg wielokąta), dla którego:

\[ Area(w) \geq \frac{4* t^2}{M^3} \geq \frac{4 * (\frac{D}{47})^2}{M^3} = |w|^2 * \frac{4}{2209 * M^3}  \]

W ten sposób dostajemy ciąg słów o długościach rosnących do nieskończoności ($|w| \geq t$, więc te długości nie są małe), co przeczy podkwadratowości funkcji Dehna.


% **** THIS ENDS THE EXAMPLES. DON'T DELETE THE FOLLOWING LINE:

\end{document}





